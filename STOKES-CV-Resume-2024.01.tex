\let\nofiles\relax


\documentclass[line,margin,10pt]{res} 

\usepackage{hyperref}
\usepackage[sfdefault]{roboto}  %% Option 'sfdefault' only if the base font of the document is to be sans serif
\usepackage[T1]{fontenc}

\oddsidemargin -0.3in
\textwidth 5.75in

%Page numbering
%\usepackage{lastpage}
%\usepackage{fancyhdr}
%\pagestyle{fancy}
%\fancyhf{} % clear existing header/footer entries
% Place Page X of Y on the right-hand
% side of the footer
%\fancyfoot[R]{Page \thepage \hspace{1pt} of \pageref{LastPage}}
%

\begin{document}
\name{Jared D. Stokes, PhD}
%\name{Jared D. Stokes, PhD \quad \quad \quad \quad \quad \quad \quad \quad  \quad UC Davis Psychiatry}
%\address{(919) 260-1824   \quad jdstokes@ucdavis.edu \quad }

\begin{resume}
\moveleft 0.5\hoffset\centerline 
{\hyperref[stokesjd@gmail.com]{stokesjd@gmail.com} \quad \quad \quad \quad  \quad \quad \quad \quad \quad \quad \quad \quad \quad \quad \quad \quad \quad \quad \quad \quad\quad \quad \quad \quad \quad \quad \quad \quad \quad \quad \quad  \quad \quad \quad \quad \quad 919-260-1841}
\moveleft 0.5\hoffset\centerline 
{\hspace{0.00cm} \hyperref[linkedin.com/in/stokesjd]{linkedin.com/in/stokesjd} \quad \quad \quad \quad \quad \quad \quad \quad \quad \quad \quad\quad \quad \quad \quad  \quad \quad \quad \quad \quad \quad \quad \quad \quad  \quad \quad \quad  \quad \quad \quad   \quad \hyperref[github.com/jdstokes]{github.com/jdstokes}}


%\section{PROFESSIONAL PROFILE} 
\begin{itemize} \itemsep -2pt
\item  []{Highly motivated research scientist with technical experience in virtual reality software development and eye-tracking. Fields of inquiry include memory, spatial navigation, attention and distraction. Over 10 years of
experience in experimental design and data analysis.}\\


%\item  []{\sl Dr. Stokes has expertise in virtual reality, eye-tracking, cognitive	
%neuroscience, human behavior, spatial navigation, and attention and disctraction
%cognitive impairments along with extensive
%experience in experimental design and data analysis.}\\
%\item Designed virtual reality tasks to investigate attention, memory and navigation
%\item Collected and analyzed behavioral and task-oriented neuroimaging datasets
%\item Applied machine learning to investigate virtual reality evoked brain activity
%\item Developed communication, teamwork and writing skills through management, research, and teaching experience in an academic setting
\end{itemize}

\section{AREAS OF EXPERTISE} 
\begin{itemize} \itemsep -2pt
\item Virtual and Augmented Reality
\item Eye-Tracking
\item Human Memory
\item Spatial Navigation
\item Attention and Distraction
\item Data Analysis
\end{itemize}

\section{SOFTWARE} 
\begin{itemize} \itemsep -2pt
\item Unity 
\item Python
\item R
\item Adobe Illustrator and Photoshop
\item Matlab
\end{itemize}

\section{EDUCATION} 
\textbf{UNIVERSITY OF CALIFORNIA, DAVIS}
\begin{itemize} \itemsep -2pt
\item  []PhD, Perception, Cognition, and Cognitive Neuroscience, 2018
\end{itemize}

\textbf{UNIVERSITY OF NORTH CAROLINA, CHAPEL HILL}
\begin{itemize} \itemsep -2pt
\item  []BS, Biology, 2005
\end{itemize}

\section{WORK HISTORY} 

\textbf{Postdoctoral Researcher } \hfill 2018-2021 \\
\emph{UC Davis Medical Center, Sacramento, CA}\\
\begin{itemize} 
\item Developed a virtual reality classroom to investigate attention and distraction under real-world settings.
\item Analyzed eye-tracking data collected from virtual reality headsets under real-world conditions.
\item Collaborated with investigators and community members to better understand and address attention and distraction disorders.
\end{itemize}

\textbf{Lab Manager } \hfill 2006-2011 \\
\emph{Duke University, Durham, NC}\\
\begin{itemize} \itemsep -2pt
\item Developed memory tasks to investigate the relationship between lab-based episodic memory and autobiographical memory, as well as tasks to investigate memory in both healthy older adults and MCI patients.
\item Hired undergraduate researchers, aided in grant duties, and managed a large cognitive neuroscience lab.
\end{itemize}


%\section{RESEARCH \& TECHNICAL EXPERIENCE} 
%\textbf{University of California, Davis, CA } \hfill 2018-present \\
%CTSC Post-Doctoral Clinical Research Training Program\\
%Mentors: Julie Schweitzer, Ph.D. \& Joy Geng, Ph.D.\\
%\\
%Affiliations:\\
%Attention, Impulsivity and Regulation (AIR) Program\\
%UC Davis MIND Institute\\
%Department of Psychiatry \& Behavioral Sciences\\
%University of California Davis School of Medicine
%\\
%Center for Mind and Brain \\
%UC Davis, CA\\
%
%\begin{itemize} \itemsep -2pt
%\item  []Selected projects:
%\item \emph{Studying attention using virtual reality classroom.} Programmed and developed a virtual reality classroom to investigate attention and distraction under real-world settings. Data collection is currently underway in phase I of an NIH funded clinical trial.
%\item \emph{Virtual reality singleton distraction task.} Programmed and developed a virtual reality version of the continuous singleton distraction task. We are currently in the data collection stage.
%\item \emph{UC Davis XR research group.} Building a virtual and augmented reality cognitive science focused organization on the UC Davis campus. Our goal is to offer educational and applied XR research opportunities to the undergraduate community.
%\end{itemize}
%
%\textbf{University of California, Davis} \hfill 2011-2018 \\
%Graduate Student under Dr. Arne Ekstrom\\
%Spatial Cognition Lab, Center for Neuroscience
%\begin{itemize} \itemsep -2pt
%\item  []Selected projects:
%	\item \emph{Investigating spatial navigation using immersive VR.} Designed, developed, and programmed navigation and spatial memory tasks to incorporate the Oculus rift and an omnidirectional treadmill. Results produced a manuscript. 
%\item \emph{Understanding the influence of prior knowledge on spatial memory formation.} Designed, developed and programmed an ensemble of VR navigation, map-assessment, and pre-training tasks aimed to investigate how prior knowledge mediates environment learning. Collected and analyzed fMRI data using pattern similarity analyses, non-parametric permutation tests, advanced image registration and 3D data visualization. Results were presented in academic conferences and produced a manuscript currently under review.
%\item \emph{Investigating representations for virtual environments within the medial temporal lobe.} Designed, developed and programmed novel, VR passive navigation tasks to assess spatial environment learning and differentiation for neuroimaging experimentation. Collected fMRI data and conducted multivariate pattern similarity and functional connectivity analyses to localize signals within the human medial temporal lobe. Results were presented in academic conferences and produced a published manuscript.
%\end{itemize} 
%
%\textbf{Duke University, Durham, NC } \hfill 2006-2011 \\
%Research specialist/Lab manager under Dr. Roberta Cabeza\\
%Center for Cognitive Neuroscience
%\begin{itemize} \itemsep -2pt
%\item  []Selected projects:
%\item \emph{Comparing lab-based to real-life memory.} Programmed and developed memory tasks to investigate the relationship between lab-based episodic memory and more ecologically valid, autobiographical memory. Collected and analyzed fMRI data. Results produced a published manuscript. 
%\item \emph{Neural correlates of attention and memory.} Programmed and developed memory tasks to investigate the role of top-down attention during memory retrieval. Collected fMRI data, programmed analysis pipelines and performed both univariate and multivariate, functional connectivity neuroimaging analyses. Results produced a published manuscript.
%\item \emph{Understanding aging and memory using fMRI}. Programmed and developed memory tasks designed to investigate memory in both healthy older adults and MCI patients. Collected fMRI data, programmed analysis pipelines and performed both structural and functional neuroimaging analyses. Results produced two published manuscripts.
%\item \emph{Neural correlates of emotion and memory.} Programmed and developed an emotion and memory oriented behavioral task. Collected and preprocessed fMRI data. Results produced two published manuscripts. 
%\end{itemize}
%
%\textbf{Tufts University, Medford, MA } \hfill 2006\\
%Research assistant, NeuroCognition of Language Lab, Department of Psychology
%\begin{itemize} \itemsep -2pt
%\item  []Developed language oriented stimulus sets, prepared participants, and administered EEG experiments designed to investigate the neural correlates of language comprehension. 
%\end{itemize}

%\section{TECHNICAL SKILLS} 
%\emph{Data analysis:} generalized linear and mixed models; dimensionality reduction; deep neural networks; support vector machines\\
%\emph{Programming languages:} MATLAB, Python, R, C\#(Unity), Bash\\
%\emph{Experimental task design:} Unity, Cogent, PsychoPy, Psychtoolbox, E-prime  \\
%\emph{Neuroimaging:} Certified 3T MRI Operator (Philips, UC Irvine; Siemens, UC Davis), SPM, FSL, ANTs, ASHS, ParaView,  MRIcro, ITK-snap \\
%\emph{Statistical packages:} Scikit-learn, TensorFlow, R SPSS, SAS\\
%\emph{General:} Git, Blender, AWS, Adobe Photoshop, Adobe Illustrator, Microsoft Word, Excel, and PowerPoint\\
%\emph{Operating systems:} Windows, Mac, and Linux operating systems\\


%\section {INTERESTS \& SPECIALIZATIONS}
%\begin{tabular}{ l l l }
%Memory & \quad \quad Navigation & \quad \quad Hippocampus\\
%Virtual reality & \quad \quad Neuroimaging & \quad \quad Neural networks\\
%Cognitive task design & \quad \quad Machine learning & \quad \quad Computational modeling\\ 
%\end{tabular}

%\section{EDUCATION} 
%\textbf{Ph.D.}, Psychology, University of California, (Degree expected Spring 2018)\\
%\emph{Dissertation}: Representations of virtual environments in the human hippocampus\\
%\textbf{B.Sc.}, University of North Carolina, Chapel Hill, NC, (2000-2005)\\
%Major: Biology (Chemistry minor)\\
%				
%\section{AWARDS / WORKSHOPS}
%\begin{itemize} \itemsep -2pt
%\item Virtual/Augmented Reality Development, Circuit Stream, Spring/Summer 2019
%\item I-Corps@NCATS Regional Short Course, UC Davis, Fall 2018
%\item TL1 Postdoctoral Clinical Research Training Program Scholar Award, Spring 2018
%\item Dissertation Fellowship, UC Davis, Winter 2017
%\item Dukes Travel Award, UC Davis, Fall 2016
%\end{itemize}
%
%\section{LEADERSHIP EXPERIENCE}
% \textbf{Teaching Assistant,} Department of Psychology, UC Davis \hfill 2011-present \\
%\emph{Courses:} Cognitive Neuroscience, Cognitive Psychology, Development of memory, Neurobiology of Learning and Memory, Research Methods, Human Perception, Human Learning and Memory, Introduction to Psychology\\
%\emph{Duties:} Contributed guest lectures, held review sessions, lead discussion sections, proctored exams, created testing materials, graded papers and exams, held regular office hours\\
%\textbf{Mentorship,} Department of Psychology, UC Davis \\
%\emph{Project virtual davis mentorship}\hfill 2016 \\
%Supervised an undergraduate researcher in the creation of realistic 3D models of buildings located in Davis, CA. The project resulted in a realistic rendering of the UC Davis Center for Neuroscience, which is currently utilized for research purposes.\\
%\emph{Thesis project mentorship}\hfill 2015-2016 \\
%Supervised three UC Davis undergraduates students during the development, data collection, data analysis and scientific communication of their senior research projects.\\

%\textbf{Guest lecturer:} \\
%\textbf{Article reviewer:} \\

\section{AWARDS AND WORKSHOPS}
\begin{itemize} \itemsep -2pt
\item Virtual/Augmented Reality Development, Circuit Stream, Spring/Summer 2019
\item I-Corps@NCATS Regional Short Course, UC Davis, Fall 2018
\item TL1 Postdoctoral Clinical Research Training Program Scholar Award, Spring 2018
\item Dissertation Fellowship, UC Davis, Winter 2017
\item Dukes Travel Award, UC Davis, Fall 2016
\end{itemize}

\section{PUBLICATIONS}

\textbf{Stokes, J.D.}, Rizzo, A., Geng, J.J., Schweitzer, J.B. (2022). Measuring Attentional Distraction in Children With ADHD Using Virtual Reality Technology With Eye-Tracking. Front Virtual Real.

Starrett, M.J., McAvan, A.S., Huffman, D.J., \textbf{Stokes, J.D.}, Kyle, C.T., Smuda, D.N., Kolarik, B.S, Laczko, J., Ekstrom, A.D.(2021) Landmarks: A solution for spatial navigation and memory experiments in virtual reality. Behav Res Methods.

Kyle, C.T., \textbf{Stokes, J.D.}, Bennett, J., Meltzer, J., Permenter, M.R., Vogt, J.A., Ekstrom, A., Barnes, C.A. (2019). Cytoarchitectonically-driven MRI atlas of nonhuman primate hippocampus: Preservation of subfield volumes in aging. Hippocampus.

\textbf{Stokes, J.D.}, Kyle, C., Huffman, D., Ekstrom, A.D. (2018). Integration of novel shape templates during human spatial navigation leads to prototype extraction, non-Euclidean environments. SSRN Electronic Journal.

Starrett, M.J., \textbf{Stokes, J.D.}, Huffman, D. Ekstrom, A.D. (2018). Learning-Dependent Evolution of Spatial Representations in Large-Scale Virtual Environments. Journal of Experimental Psychology: Learning, Memory, and Cognition.

Monge, Z.A., Wing, E. A., \textbf{Stokes J.}, Cabeza, R. (2017). Search and Recovery of Autobiographical and Laboratory Memories: Shared and Distinct Neural Components. Neuropsychologia.

Bouffard, N., \textbf{Stokes, J.}, Kramer, H., Ekstrom, A. (2017). Temporal encoding strategies result in boosts to final free recall performance comparable to spatial ones. Memory \& Cognition.

Lieberman, J.S., Kyle, C. T., Schedlbauer, A., \textbf{Stokes, J.D.}, Ekstrom, A. D. (2017). A tale of two temporal coding strategies: Common and dissociable brain regions involved in recency vs. associative temporal order retrieval strategies. Journal of Cognitive Neuroscience.

Kyle, C. T., \textbf{Stokes, J.D.}, Lieberman, J. S., Hassan, A. S., Ekstrom, A. D. (2015). Successful retrieval of competing spatial environments in humans involves hippocampal pattern separation mechanisms. eLife, 4. 

\textbf{Stokes, J.D.}, Kyle, C., Ekstrom, A. D. (2015). Complementary Roles of Human Hippocampal Subfields in Differentiation and Integration of Spatial Context. Journal of Cognitive Neuroscience, 27(3), 546-559. 

Dolcos, F., Iordan, A. D., Kragel, J., \textbf{Stokes, J.D.}, Campbell, R., McCarthy, G., Cabeza, R. (2013). Neural correlates of opposing effects of emotional distraction on working memory and episodic memory: an event-related FMRI investigation. Frontiers in Psychology, 4, 293. 

Shafer, A. T., Matveychuk, D., Penney, T., O'Hare, A. J., \textbf{Stokes, J.D.}, Dolcos, F. (2012). Processing of emotional distraction is both automatic and modulated by attention: evidence from an event-related fMRI investigation. Journal of Cognitive Neuroscience, 24(5), 1233?1252.
 		
Hayes, S.M., Buchler, N., \textbf{Stokes, J.D.}, Kragel, J., Cabeza, R. (2011). Neural correlates of confidence during item recognition and source memory retrieval: Evidence for both dual-process and strength memory theories. Journal of Cognitive Neuroscience.
	
Cabeza, R., Mazuz, M., \textbf{Stokes, J.D.}, Kragel, J., Woldorff, W, Ciaramelli, E., Olson, I., Moscovitch, M. (2011). Overlapping Parietal Activity in Memory and Perception: Evidence for the Attention to Memory (AtoM) Model. Journal of Cognitive Neuroscience, 23, 3209-3217.
	
Dennis, N., Browndyke, J., \textbf{Stokes, J.D.}, Need, A., Burke, J., Welsh-Bohmer, K., Cabeza, R. (2010) Temporal lobe functional activity and connectivity in young adult APOE e4 carriers. Alzheimer's \& Dementia.

\section{SELECTED PRESENTATIONS} 
\textbf{Stokes, J.D.}(2019) Enhancing attention in children using a virtual classroom. CTSC 15th Annual Scholar Symposium, UC Davis.

%Kyle, C.T,\textbf{Stokes, J.D.}, Meltzer, J., Permenter, M.R, Vogt, J.A, Ekstrom, A.D., Barnes, C.A.(2019) Estimation of non-rigid warps during 3D serial-section histology reconstruction optimization increases accuracy. Society for Neuroscience Society Abstracts.

\textbf{Stokes, J.D.}, Kyle, C., Huffman, D., Ekstrom, A.D. (2018) Human hippocampal representations of novel, irregular environments. International Conference on Learning \& Memory, UC Irvine.

%Starrett, M.J., \textbf{Stokes, J.D.}, Kreylos, O., Ekstrom, A. D., (2016) Navigation in virtual reality with vestibular and proprioceptive input diminishes orientation-dependent spatial representations. Society for Neuroscience Society Abstracts.

%Kyle, C., Bennett, J. L., \textbf{Stokes, J.D.}, Permenter, M. R., Vogt, J. A., Ekstrom, A. D., Barnes, C. A. (2016) Histology informed probabilistic hippocampal atlases of young and old rhesus macaques. Society for Neuroscience Society Abstracts.
%
%Borders, A., \textbf{Stokes, J.D.}, Kyle, C., Ekstrom, A., Yonelinas, A. (2015) High-resolution hippocampal activation patterns predict memory precision. Society for Neuroscience Society Abstracts.

\textbf{Stokes, J.D.}, Kyle, C., Ekstrom, A. (2015) Integration of familiar and novel spatial templates in episodic memory. Society for Neuroscience Society Abstracts.

%Bouffard, N., \textbf{Stokes, J.D.}, Kyle, C., Lieberman, J., Ekstrom, A. (2015) Temporal encoding strategies produce comparable boosts in free recall performance to spatial encoding strategies. Society for Neuroscience Abstracts.
%
%Lieberman, J., \textbf{Stokes, J.D.}, Kyle, C., Ekstrom, A. (2015) A tale of two temporal retrieval strategies: Dynamic expression of temporal sequence retrieval. Society for Neuroscience Abstracts.
%
%Kyle, C., \textbf{Stokes, J.D.}, Ekstrom, A. (2014) Properties of spatial contextual representation within the human hippocampus during episodic memory retrieval. Society for Neuroscience Abstracts.

\textbf{Stokes, J.D.}, Kyle, C., Ekstrom, A. (2014) Dissociable roles of human hippocampal subfields CA3/DG and CA1 during processing of spatial context. Society for Neuroscience Abstracts.

%\textbf{Stokes, J.D.}, Kyle, C., Ekstrom, A. (2014) Dissociable codes within the human hippocampal subfields during spatial context processing. Bay Area Memory Meeting Abstracts.

\textbf{Stokes, J.D.}, Ekstrom, A. (2012) Representational similarity in CA3/DG tracks changes in spatial context. Cognitive Neuroscience Society Abstracts.
 	
%Smuda, D., Kyle, C., \textbf{Stokes, J.D.}, Ekstrom, A. (2012) Role of hippocampal subregions in disambiguating elements of temporal vs. spatial context in episodic memory. Cognitive Neuroscience Society Abstracts.
% 	
\textbf{Stokes, J.D.}, Mazuz, Y., Daselaar, S., Moscovitch, M., Cabeza, R. (2011) Similarities and differences between the neural mechanisms of episodic and autobiographical memory recall. Cognitive Neuroscience Society Abstracts.
 	
%Hayes, S., Buchler, N., \textbf{Stokes, J.D.}, J, Kragel J.,  Cabeza, R. (2010) Recollection orientation, retrieval success, and task difficulty: The role of prefrontal cortex and posterior parietal cortex during source and item memory. Cognitive Neuroscience Society Abstract.
% 	
%Tomlinson, S., Kragel, J., \textbf{Stokes, J.D.}, Dolcos, F., McCarthy, G., Cabeza, R. (2008). Role of individual differences in the response to emotional distraction: An event-related fMRI investigation. Supplement of Journal of Cognitive Neuroscience Abstracts.
% 	
%Dolcos, F., \textbf{Stokes, J.D.}, Kragel, J., Ritchey, M. Tsukiura, T. McCarthy, G., Cabeza, R. (2007). Neural correlates of opposing modulation of emotion on short- vs. long-term memory processes: An event-related fMRI investigation. Society for Neuroscience Abstracts.

\end{resume}
\end{document}