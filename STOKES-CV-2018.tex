\documentclass[line,margin,10pt]{res} 

\oddsidemargin -0.25in
\textwidth 6in

\begin{document}

\name{Jared D. Stokes, PhD candidate \quad \quad \quad \quad \quad \quad \quad \quad \quad \quad \quad \quad  \quad \quad UC Davis Psychology}
\address{(919) 260-1824}
\address {jdstokes@ucdavis.edu}

\begin{resume}

\section{SUMMARY} 
\begin{itemize} \itemsep -2pt
\item  []{\sl Doctoral candidate in psychology studying human memory, specifically, characterizing neural representations of individual memories associated with real-life behavior}\\
\item Designed virtual reality tasks to investigate memory and navigation
\item Collected, preprocessed and analyzed behavioral and task-oriented neuroimaging datasets
\item Applied machine learning to investigate virtual reality evoked brain activity
\item Developed communication, teamwork and writing skills through management, research, and teaching experience in an academic setting
\end{itemize}

\section{RESEARCH \& TECHNICAL EXPERIENCE} 
\textbf{University of California, Davis} \hfill 2011-present \\
Graduate Student under Dr. Arne Ekstrom\\
Spatial Cognition Lab, Center for Neuroscience
\begin{itemize} \itemsep -2pt
\item  []Selected projects:
\item \emph{Using artificial neural networks to investigate real-world navigation and explain brain-activity patterns.} Classified spatial environments during a memory retrieval task using functional magnetic resonance imaging (fMRI) data (resulting in a published manuscript). Programmed analysis pipeline to extend this approach and used an artificial neural network to decode environments during a realistic navigation paradigm using task evoked fMRI signals within the hippocampus. Results were presented in the UC Davis, Advanced Topics in Neuroimaging course, Spring 2017.
\item \emph{Investigating spatial navigation using immersive VR.} Designed and programmed navigation and spatial memory tasks to incorporate the Oculus rift and an omnidirectional treadmill to assess spatial memory formation under real-life circumstances. Results produced a manuscript currently under review. 
\item \emph{Understanding the influence of prior knowledge on spatial memory formation.} Designed, developed and programmed an ensemble of VR navigation, map-assessment, and pre-training tasks aimed to investigate how prior knowledge mediates environment learning. Collected and analyzed fMRI data using pattern similarity analyses, bidimensional regression, non-parametric permutation tests, advanced image registration and 3D data visualization. Results were presented in academic conferences and produced a manuscript currently under review.
\item \emph{Investigating representations for virtual environments within the medial temporal lobe.} Designed, programmed and developed novel, VR passive navigation tasks to assess spatial environment learning and differentiation for neuroimaging experimentation. Collected fMRI data and conducted multivariate pattern similarity and functional connectivity analyses to localize signals within the human medial temporal lobe. Results were presented in academic conferences and produced a published manuscript.
\end{itemize} 

\textbf{Duke University, Durham, NC } \hfill 2006-2011 \\
Research specialist/Lab manager under Dr. Roberta Cabeza\\
Center for Cognitive Neuroscience
\begin{itemize} \itemsep -2pt
\item  []Selected projects:
\item \emph{Comparing lab-based to real-life memory.} Programmed and developed memory tasks to investigate the relationship between lab-based episodic memory and more ecologically valid, autobiographical memory. Collected and analyzed fMRI data. Results produced a published manuscript. 
\item \emph{Neural correlates of attention and memory.} Programmed and developed memory tasks to investigate the role of top-down attention during memory retrieval. Collected fMRI data, programmed analysis pipelines and performed both univariate and multivariate, functional connectivity neuroimaging analyses. Results produced a published manuscript.
\item \emph{Understanding aging and memory using fMRI}. Programmed and developed memory tasks designed to investigate memory in both healthy older adults and MCI patients. Collected fMRI data, programmed analysis pipelines and performed both structural and functional neuroimaging analyses. Results produced two published manuscripts.
\item \emph{Neural correlates of emotion and memory.} Programmed and developed an emotion and memory oriented behavioral task. Collected and preprocessed fMRI data. Results produced two published manuscripts. 
\end{itemize}

\textbf{Tufts University, Medford, MA } \hfill 2006\\
Research assistant, NeuroCognition of Language Lab, Department of Psychology
\begin{itemize} \itemsep -2pt
\item  []Developed language oriented stimulus sets, prepared participants, and administered EEG experiments designed to investigate the neural correlates of language comprehension. 
\end{itemize}

\section{TECHNICAL SKILLS} 
\emph{Data analysis:} generalized linear and mixed models; dimensionality reduction; deep neural networks; support vector machines\\
\emph{Programming languages:} MATLAB, Python, R, Javascript(Unity), C\#(Unity), Bash\\
\emph{Experimental task design:} Unity, Cogent, PsychoPy, Psychtoolbox, E-prime  \\
\emph{Neuroimaging:} Certified 3T MRI Operator (Philips, UC Irvine; Siemens, UC Davis), SPM, FSL, ANTs, ASHS, ParaView,  MRIcro, ITK-snap \\
\emph{Statistical packages:} scikit-learn, TensorFlow, SPSS, SAS\\
\emph{General:} Git, Blender, AWS, Adobe Photoshop, Adobe Illustrator, Microsoft Word, Excel, and PowerPoint\\
\emph{Operating systems:} Windows, Mac, and Linux operating systems\\


%\section {INTERESTS \& SPECIALIZATIONS}
%\begin{tabular}{ l l l }
%Memory & \quad \quad Navigation & \quad \quad Hippocampus\\
%Virtual reality & \quad \quad Neuroimaging & \quad \quad Neural networks\\
%Cognitive task design & \quad \quad Machine learning & \quad \quad Computational modeling\\ 
%\end{tabular}

\section{EDUCATION} 
\textbf{Ph.D.}, Psychology, University of California, (Degree expected Fall 2018)\\
\emph{Dissertation}: Representations of virtual environments in the human hippocampus\\
\begin{itemize} \itemsep -2pt
\item Award recipient, Dissertation Fellowship, UC Davis, Winter 2017
\item	Award recipient, Dukes Travel Award, UC Davis, Fall 2016
\end{itemize}

\textbf{B.Sc.}, University of North Carolina, Chapel Hill, NC, (2000-2005)\\
Major: Biology (Chemistry minor)\\
\begin{itemize} \itemsep -2pt
\item	Rehabilitation Unit Volunteer, Long-term rehab clinic; John Umstead Hospital, 2004
\item	Field Research Assistant, Dr. K.A.I. Nekaris, Sri Lanka, 2004
\item	Student Project, La Suerte Biological Field Station, Costa Rica, 2003
\item	Research Assistant, Clemmons Lab, UNC Department of Medicine, 2001-2003
\end{itemize}								

\section{LEADERSHIP EXPERIENCE}
 \textbf{Teaching Assistant,} Department of Psychology, UC Davis \hfill 2011-present \\
\emph{Courses:} Cognitive Neuroscience, Cognitive Psychology, Development of memory, Neurobiology of Learning and Memory, Research Methods, Human Perception, Human Learning and Memory, Introduction to Psychology\\
\emph{Duties:} Contributed guest lectures, held review sessions, lead discussion sections, proctored exams, created testing materials, graded papers and exams, held regular office hours\\
\textbf{Mentorship,} Department of Psychology, UC Davis \\
\emph{Project virtual davis mentorship}\hfill 2016 \\
Supervised an undergraduate researcher in the creation of realistic 3D models of buildings located in Davis, CA. The project resulted in a realistic rendering of the UC Davis Center for Neuroscience, which is currently utilized for research purposes.\\
\emph{Thesis project mentorship}\hfill 2015-2016 \\
Supervised three UC Davis undergraduates students during the development, data collection, data analysis and scientific communication of their senior research projects.\\

%\textbf{Guest lecturer:} \\
%\textbf{Article reviewer:} \\

\section{PUBLICATIONS}

\textbf{Stokes, J.D.}, Kyle, C., Huffman, D., Ekstrom, A.D. (under review) Integration of novel shape templates during human spatial navigation leads to prototype extraction, non-Euclidean environments.

Starrett, M.J., \textbf{Stokes, J.D.}, Ekstrom, A.D. (under review). Learning-Dependent Evolution of Spatial Representations in Large-Scale Virtual Environments.

Monge, Z.A., Wing, E. A., \textbf{Stokes J.}, Cabeza, R. (2017). Search and Recovery of Autobiographical and Laboratory Memories: Shared and Distinct Neural Components. Neuropsychologia.

Bouffard, N., \textbf{Stokes, J.}, Kramer, H., Ekstrom, A. (2017). Temporal encoding strategies result in boosts to final free recall performance comparable to spatial ones. Memory \& Cognition.

Kyle, C.T., \textbf{Stokes, J.D.}, Bennett, J., Meltzer, J., Permenter, M.R., Vogt, J.A., Ekstrom, A., Barnes, C.A. (2017) Cytoarchitectonically-driven MRI atlas of nonhuman primate hippocampus:  preservation of subfield volumes in aging. Hippocampus. 

Lieberman, J.S., Kyle, C. T., Schedlbauer, A., \textbf{Stokes, J.D.}, Ekstrom, A. D. (2017). A tale of two temporal coding strategies: Common and dissociable brain regions involved in recency vs. associative temporal order retrieval strategies. Journal of Cognitive Neuroscience.

Kyle, C. T., \textbf{Stokes, J.D.}, Lieberman, J. S., Hassan, A. S., Ekstrom, A. D. (2015). Successful retrieval of competing spatial environments in humans involves hippocampal pattern separation mechanisms. eLife, 4. 

\textbf{Stokes, J.D.}, Kyle, C., Ekstrom, A. D. (2015). Complementary Roles of Human Hippocampal Subfields in Differentiation and Integration of Spatial Context. Journal of Cognitive Neuroscience, 27(3), 546-559. 

Dolcos, F., Iordan, A. D., Kragel, J., \textbf{Stokes, J.D.}, Campbell, R., McCarthy, G., Cabeza, R. (2013). Neural correlates of opposing effects of emotional distraction on working memory and episodic memory: an event-related FMRI investigation. Frontiers in Psychology, 4, 293. 

Shafer, A. T., Matveychuk, D., Penney, T., O'Hare, A. J., \textbf{Stokes, J.D.}, Dolcos, F. (2012). Processing of emotional distraction is both automatic and modulated by attention: evidence from an event-related fMRI investigation. Journal of Cognitive Neuroscience, 24(5), 1233?1252.
 		
Hayes, S.M., Buchler, N., \textbf{Stokes, J.D.}, Kragel, J., Cabeza, R. (2011). Neural correlates of confidence during item recognition and source memory retrieval: Evidence for both dual-process and strength memory theories. Journal of Cognitive Neuroscience.
	
Cabeza, R., Mazuz, M., \textbf{Stokes, J.D.}, Kragel, J., Woldorff, W, Ciaramelli, E., Olson, I., Moscovitch, M. (2011). Overlapping Parietal Activity in Memory and Perception: Evidence for the Attention to Memory (AtoM) Model. Journal of Cognitive Neuroscience, 23, 3209-3217.
	
Dennis, N., Browndyke, J., \textbf{Stokes, J.D.}, Need, A., Burke, J., Welsh-Bohmer, K., Cabeza, R. (2010) Temporal lobe functional activity and connectivity in young adult APOE e4 carriers. Alzheimer's \& Dementia.

\section{CONFERENCE PROCEEDINGS} 

\textbf{Stokes, J.D.}, Kyle, C., Huffman, D., Ekstrom, A.D. (2018) Human hippocampal representations of novel, irregular environments. International Conference on Learning \& Memory, UC Irvine.

Starrett, M.J., \textbf{Stokes, J.D.}, Kreylos, O., Ekstrom, A. D., (2016) Navigation in virtual reality with vestibular and proprioceptive input diminishes orientation-dependent spatial representations. Society for Neuroscience Society Abstracts.

Kyle, C., Bennett, J. L., \textbf{Stokes, J.D.}, Permenter, M. R., Vogt, J. A., Ekstrom, A. D., Barnes, C. A. (2016) Histology informed probabilistic hippocampal atlases of young and old rhesus macaques. Society for Neuroscience Society Abstracts.

Borders, A., \textbf{Stokes, J.D.}, Kyle, C., Ekstrom, A., Yonelinas, A. (2015) High-resolution hippocampal activation patterns predict memory precision. Society for Neuroscience Society Abstracts.

\textbf{Stokes, J.D.}, Kyle, C., Ekstrom, A. (2015) Integration of familiar and novel spatial templates in episodic memory. Society for Neuroscience Society Abstracts.

Bouffard, N., \textbf{Stokes, J.D.}, Kyle, C., Lieberman, J., Ekstrom, A. (2015) Temporal encoding strategies produce comparable boosts in free recall performance to spatial encoding strategies. Society for Neuroscience Abstracts.

Lieberman, J., \textbf{Stokes, J.D.}, Kyle, C., Ekstrom, A. (2015) A tale of two temporal retrieval strategies: Dynamic expression of temporal sequence retrieval. Society for Neuroscience Abstracts.

Kyle, C., \textbf{Stokes, J.D.}, Ekstrom, A. (2014) Properties of spatial contextual representation within the human hippocampus during episodic memory retrieval. Society for Neuroscience Abstracts.

\textbf{Stokes, J.D.}, Kyle, C., Ekstrom, A. (2014) Dissociable roles of human hippocampal subfields CA3/DG and CA1 during processing of spatial context. Society for Neuroscience Abstracts.

\textbf{Stokes, J.D.}, Kyle, C., Ekstrom, A. (2014) Dissociable codes within the human hippocampal subfields during spatial context processing. Bay Area Memory Meeting Abstracts.

\textbf{Stokes, J.D.}, Ekstrom, A. (2012) Representational similarity in CA3/DG tracks changes in spatial context. Cognitive Neuroscience Society Abstracts.
 	
Smuda, D., Kyle, C., \textbf{Stokes, J.D.}, Ekstrom, A. (2012) Role of hippocampal subregions in disambiguating elements of temporal vs. spatial context in episodic memory. Cognitive Neuroscience Society Abstracts.
 	
\textbf{Stokes, J.D.}, Mazuz, Y., Daselaar, S., Moscovitch, M., Cabeza, R. (2011) Similarities and differences between the neural mechanisms of episodic and autobiographical memory recall. Cognitive Neuroscience Society Abstracts.
 	
Hayes, S., Buchler, N., \textbf{Stokes, J.D.}, J, Kragel J.,  Cabeza, R. (2010) Recollection orientation, retrieval success, and task difficulty: The role of prefrontal cortex and posterior parietal cortex during source and item memory. Cognitive Neuroscience Society Abstract.
 	
Tomlinson, S., Kragel, J., \textbf{Stokes, J.D.}, Dolcos, F., McCarthy, G., Cabeza, R. (2008). Role of individual differences in the response to emotional distraction: An event-related fMRI investigation. Supplement of Journal of Cognitive Neuroscience Abstracts.
 	
Dolcos, F., \textbf{Stokes, J.D.}, Kragel, J., Ritchey, M. Tsukiura, T. McCarthy, G., Cabeza, R. (2007). Neural correlates of opposing modulation of emotion on short- vs. long-term memory processes: An event-related fMRI investigation. Society for Neuroscience Abstracts.

\end{resume}
\end{document}